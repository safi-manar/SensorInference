\documentclass{sig-alternate-10pt}

% Instance space-saver: --cpk
%\usepackage{times}

\usepackage{url}
\usepackage{color}
\usepackage{xspace}

\usepackage{booktabs} % To thicken table lines

\usepackage{multirow}
\usepackage{epsfig,endnotes}
\usepackage{cite}
\usepackage{pbox}

\usepackage[hidelinks]{hyperref}
\def\UrlBreaks{\do\/\do-}

\newcommand\nv[1]{\textcolor{red}{[#1 --NV]}}
\newcommand\ir[1]{\textcolor{red}{[#1 --IR]}}
\newcommand\pw[1]{\textcolor{red}{[#1 --PW]}}
\newcommand\se[1]{\textcolor{blue}{[#1 --SE]}}
\newcommand\ck[1]{\textcolor{red}{[#1 --CK]}}

\newcommand\hs[1]{ICSI Haystack}


\linepenalty=200
\newenvironment{packed_enum}{
\begin{enumerate}
  \setlength{\itemsep}{1pt}
  \setlength{\parskip}{0pt}
  \setlength{\parsep}{0pt}
}{\end{enumerate}}

\newenvironment{packed_item}{
\begin{itemize}
  \setlength{\itemsep}{1pt}
  \setlength{\parskip}{0pt}
  \setlength{\parsep}{0pt}
}{\end{itemize}}

\providecommand{\etal}{\emph{et al.}\xspace}

\begin{document}


\title{TBD}


\numberofauthors{1}
\author{
%\small
\alignauthor
    Manar Safi$^1$, Irwin Reyes$^2$, Serge Egelman$^{1,2}$ \\ \vspace{2mm}
    \affaddr{{$^1$}UC Berkeley, {$^2$}ICSI} \\
}

\maketitle
\begin{abstract}

Modern smartphones come equipped with a suite of high-fidelity motion and environmental
sensors: accelerometers, gyroscopes, and light sensors, among many others. In order to
preserve user privacy, the dominant mobile platforms, Android and iOS, place restrictions
on certain device resources. Apps are required to ask the user for permission before accessing
geolocation or phone calling capabilities, for instance. However, these access controls
don't extend to environmental and motion sensors, as these sensors are considered "benign,"
so apps are able to collect measurements from these data sources without informing
the user of this usage.

In this work, we consider that smartphones are highly personal devices, typically used
by just one person, and remains in very close proximity to -- if not on the person of -- that sole user for much of
the phone's operational life. We explore the feasibility of using this unrestricted 
"benign" sensor data to infer sensitive or otherwise surprising details about the user,
such as their daily schedules, income levels, and relationship status. We discuss our
method of sensor data collection, analysis, and validation with user-reported ground
truth to build a model to infer user information.

\end{abstract}

\section{Introduction}

Foo bar~\cite{sensor-types}

\section{Related Work}

Foo bar

\section{Methodology}

\subsection{Data collection}

Foo bar

\subsection{Survey}

Foo bar

\section{Preliminary Results}

Foo bar

\section{Conclusions}

Foo bar

%\newpage
\footnotesize
\bibliographystyle{abbrv}
\bibliography{paper}  

\end{document}



